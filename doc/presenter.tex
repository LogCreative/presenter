%% ------------------------------------------------------------------------
%% Copyright (C) 2023 Log Creative <logcreative@outlook.com>
%%
%% This work may be distributed and/or modified under the
%% conditions of the LaTeX Project Public License, either version 1.3
%% of this license or (at your option) any later version.
%% The latest version of this license is in
%%   http://www.latex-project.org/lppl.txt
%% and version 1.3 or later is part of all distributions of LaTeX
%% version 2005/12/01 or later.
%%
%% This work has the LPPL maintenance status `maintained'.
%% 
%% The Current Maintainer of this work is Log Creative.
%% ------------------------------------------------------------------------

\documentclass{l3doc}
\EnableCrossrefs
\RecordChanges
\SetupDoc{reportchangedates}

\NewDocumentCommand{\presenter}{}{\textsc{Presenter}}
\title{\presenter: A \LaTeX{} presentation framework with seamless migration}
\author{Log Creative}
\date{2023-07-07 \quad v0.5.0}

\usepackage{array}
\usepackage{float}
\usepackage{booktabs}
\usepackage{l3draw}

% Preview test PDF, should use `l3build check' first before `l3build doc'.
\usepackage{graphicx}
\NewDocumentCommand{\previewtest}{O{} m}{
  \IfFileExists{#2.pdf}%
    {
      \begingroup
        \setlength{\fboxsep}{0pt}
        \framebox{%
          \includegraphics[width=0.5\linewidth,#1]{#2.pdf}%
        }
      \endgroup
    }
    {
      \includegraphics[width=0.5\linewidth,height=0.28125\linewidth,draft]{#2.pdf}
      \PackageWarning{presenter.pdf}{`#2.pdf' is missing, \MessageBreak
      please run `l3build check' before \MessageBreak
      typesetting the document}
    }
}

% Make adjustment to TOC
\makeatletter
\renewcommand*\l@subsection{\@dottedtocline{2}{1.5em}{2.8em}}
\renewcommand*\l@subsubsection{\@dottedtocline{3}{4.3em}{3.7em}}
\makeatother

\begin{document}

\maketitle

\begin{abstract}
  \presenter{} is a \LaTeX{} document package adapted to regular \LaTeX{}
  syntaxes and designed for making minimalist presentations, written in
  \LaTeX3.
\end{abstract}

\tableofcontents

\begin{documentation}

  \section{Introduction}

  With the popularity of GPT-4 and its related technology, it is now possible to
  generate presentations by just entering some prompts or summarizing an article.
  In \LaTeX{} world, \presenter{} is such a package to create presentations from
  regular \LaTeX{} documents without many modifications, which could be natively
  run by any latest \TeX{} distributions.

  Compared to other existing packages, \presenter{} won't need any additional
  \env{frame} or \env{slide} environments to split different pages, but uses
  sectioning commands as the natural way to split different points.
  This package can not summarize the contents yet but it is promising to have
  such a function in other ways in the future.

  \subsection{Seamless Migration}

  Users should feel seamless on migrating from document class like \cls{article},
  \cls{report} to \cls{presenter} without modifying many structures.

  \subsection{Stay On Top}

  Sectioning information will stay on the top of the slides.

  \subsection{Minimal Design}

  The minimal design is for catering to the current trend without introducing
  too many complexities.

  \section{Related Works}

  \begin{description}
    \item[\href{https://github.com/josephwright/beamer}{\cls{beamer}}]

      One of the most popular presentation document classes in the current
      \LaTeX{} world.

    \item[\href{https://github.com/Liam0205/easy_slides}{\pkg{easy-slides}}]

      Entirely based on \cls{article} class, to create presentations.

    \item[\href{https://github.com/LogCreative/AutoBeamer}{\pkg{AutoBeamer}}]

      Origin of the idea. Provides a package to redefine some sectioning commands
      for basic translation from \cls{article} class to \cls{beamer} class.
      Provides a web page to convert Markdown document to \cls{beamer} class code.

  \end{description}

  \section{Basics}

  Using \cls{presenter} is very simple, just use swap the line of \verb"\documentclass{article}"
  into \verb"\documentclass{presenter}" and you are almost done.

  \begin{verbatim}
    \documentclass{presenter}
    \begin{document}
      \section{Get Started}
      Hello, world!
    \end{document}
  \end{verbatim}

  \begin{function}[added=2023-02-04]{article, report}
    However, if you want to use \tn{chapter} level, pass the \verb"report"
    option to the document class \cls{presenter}. The default is \verb"article"
    based.
  \end{function}

  \begin{texnote}
    It sounds not okay to use \cls{book} class as the base document class, since
    they are typically pretty long. But if you want to use it as the base class,
    or any other document class, use the package \pkg{presenter} instead.
    \begin{verbatim}
      \documentclass{book}
      \usepackage{presenter}
      \begin{document}
        \chapter{Beginning of Everything}
        This is the beginning of everything.
      \end{document}
    \end{verbatim}
  \end{texnote}

  \section{Templates}

  \subsection{Decomposition}

  \presenter{} page can be divided into two major layers: \texttt{background}
  and \texttt{foreground}. There are several components inside each layer to
  composite the page, as is shown in Figure~\ref{fig:tpldecomp}. These layers
  are rendered in this order to overlay one after another.

  \begin{figure}
  \ExplSyntaxOn
  \draw_begin:
  \hbox_set:Nn \l_tmpa_box { \fbox{\texttt{foreground/sectioning}} }
  \draw_box_use:N \l_tmpa_box
  \draw_transform_shift:n { ( -5cm , -6.4cm ) }
  \draw_transform_xscale:n { 0.7 }
  \draw_transform_yslant:n { 0.3 }
  \hbox_set:Nn \l_tmpa_box { \previewtest[page=1]{decomp} }
  \draw_box_use:N \l_tmpa_box
  \draw_transform_xshift:n { 4cm }
  \hbox_set:Nn \l_tmpa_box { \previewtest[page=2]{decomp} }
  \draw_box_use:N \l_tmpa_box
  \draw_transform_xshift:n { 4cm }
  \hbox_set:Nn \l_tmpa_box { \previewtest[page=3]{decomp} }
  \draw_box_use:N \l_tmpa_box
  \draw_end:
  \ExplSyntaxOff
  \caption{Template Decomposition}
  \label{fig:tpldecomp}
  \end{figure}

  \begin{description}
    \item[\texttt{background}] currently contains three components: 
      \texttt{canvas}, \texttt{headline}, and \texttt{footline}.
      \begin{description}
        \item[\texttt{background/canvas}] The bottom layer.
        \item[\texttt{background/headline}] The headline background.
          If there is no headline on the page, then this layer won't get rendered.
        \item[\texttt{background/footline}] The footline background.
          If there is no footline on the page, then this layer won't get rendered. 
      \end{description}
    \item[\texttt{foreground}] currently contains four components:
      \texttt{sectioning}, \texttt{headline}, \texttt{footline} and \texttt{canvas}.
      \begin{description}
        \item[\texttt{foreground/sectioning}] A special layer handling the
          sectioning logic during processing the body. The information processed
          in this layer could be used globally (both \texttt{background} and
          \texttt{foreground}).
        \item[\texttt{foreground/headline}] The headline text.
        \item[\texttt{foreground/footline}] The footline text.
        \item[\texttt{foreground/canvas}] The top layer. 
      \end{description}
  \end{description}

  \presenter{} provides some built-in templates for changing the visual
  presentation of these components. The following command provides the
  interfaces for loading or editing them. You could visit the \texttt{testfiles/}
  folder for the code example of using these templates.

  \subsection{Load Templates}

  \begin{function}[added=2023-03-17,updated=2023-07-14]{\LoadPresenterBackground}
    \begin{syntax}
      \cs{LoadPresenterBackground} \oarg{component comma list} \marg{style}
    \end{syntax}
    Load presenter background style with style file \file{pretbg-<style>.sty}.
    The background style selection could be viewed in Section \ref{sec:pretbg}.
    You could specify the optional argument \meta{component comma list} to
    modify one or more components. \meta{component} could be one of
    \verb"canvas", \verb"headline" and \verb"footline". This command could only
    be used in the preamble.
  \end{function}

  \begin{function}[added=2023-03-17,updated=2023-07-14]{\LoadPresenterForeground}
    \begin{syntax}
      \cs{LoadPresenterForeground} \oarg{component comma list} \marg{style}
    \end{syntax}
    Load presenter background style with style file \file{pretfg-<style>.sty}.
    The foreground style selection could be viewed in Section \ref{sec:pretfg}.
    You could specify the optional argument \meta{component comma list} to
    modify one or more components. \meta{component} could be one of
    \verb"headline", \verb"sectioning", \verb"footline" and \verb"canvas". This
    command could only be used in the preamble.
  \end{function}

  \subsection{Edit Templates}

  \begin{function}[added=2023-07-11]{\EditPresenterBackground}
    \begin{syntax}
      \cs{EditPresenterBackground} \marg{component} \marg{key-value list}
    \end{syntax}
    Edit the \meta{component} by the changed properties indicated by the
    \meta{key-value list} (\textit{e.g.} \verb"stroke=true,stroke-color=blue"
    for background style \texttt{block}).
    
    The style is indicated by the last \cs{LoadPresenterBackground}. Different
    styles may have different property keys, which could be referred to in
    Section \ref{sec:pretbg}. The editing is invalid after the next
    \cs{LoadPresenterBackground} since the loading will initialize the template
    with the default values of that style.
  \end{function}

  \begin{function}[added=2023-07-11]{\EditPresenterForeground}
    \begin{syntax}
      \cs{EditPresenterForeground} \marg{component} \marg{key-value list}
    \end{syntax}
    Edit the \meta{component} by the changed properties indicated by the
    \meta{key-value list} (\textit{e.g.}
    \verb"parent-style=\Large,child-style=\normalsize" for foreground style
    \texttt{dual}).
    
    The style is indicated by the last \cs{LoadPresenterForeground}. Different
    styles may have different property keys, which could be referred to in
    Section \ref{sec:pretfg}. The editing is invalid after the next
    \cs{LoadPresenterForeground} since the loading will initialize the template
    with the default values of that style.
  \end{function}

\end{documentation}

\CheckSum{0}

\DocInput{presenter.dtx}

\DocInput{pretsec.dtx}

\DocInput{prettpl.dtx}

\DocInput{pretbg.dtx}

\DocInput{pretfg.dtx}

\PrintChanges

\PrintIndex

\end{document}
