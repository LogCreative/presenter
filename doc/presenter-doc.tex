%% ------------------------------------------------------------------------
%% Copyright (C) 2023 Log Creative <logcreative@outlook.com>
%%
%% This work may be distributed and/or modified under the
%% conditions of the LaTeX Project Public License, either version 1.3
%% of this license or (at your option) any later version.
%% The latest version of this license is in
%%   http://www.latex-project.org/lppl.txt
%% and version 1.3 or later is part of all distributions of LaTeX
%% version 2005/12/01 or later.
%%
%% This work has the LPPL maintenance status `maintained'.
%% 
%% The Current Maintainer of this work is Log Creative.
%% ------------------------------------------------------------------------

\documentclass{l3doc}
% \DisableImplementation
\EnableCrossrefs
\RecordChanges
\SetupDoc{reportchangedates}

\NewDocumentCommand{\presenter}{}{\textsc{Presenter}}
\title{\presenter: A \LaTeX{} presentation framework with seamless migration}
\author{Zilong Li (Log Creative)}
\date{2023-07-18 \quad v0.6.0}

\usepackage{array}
\usepackage{float}
\usepackage{booktabs}
\usepackage{l3draw}

% Preview test PDF, should use `l3build check' first before `l3build doc'.
\usepackage{graphicx}
\NewDocumentCommand{\previewtest}{O{} m}{
  \IfFileExists{#2.pdf}%
    {
      \begingroup
        \setlength{\fboxsep}{0pt}
        \framebox{%
          \includegraphics[width=0.3\paperwidth,#1]{#2.pdf}%
        }
      \endgroup
    }
    {
      \includegraphics[width=0.3\paperwidth,height=0.16875\paperwidth,draft]{#2.pdf}
      \PackageWarning{presenter.pdf}{`#2.pdf' is missing, \MessageBreak
      please run `l3build check' before \MessageBreak
      typesetting the document}
    }
}

% Make adjustment to TOC
\makeatletter
\renewcommand*\l@subsection{\@dottedtocline{2}{1.5em}{2.8em}}
\renewcommand*\l@subsubsection{\@dottedtocline{3}{4.3em}{3.7em}}
\makeatother

\begin{document}

\maketitle

\begin{abstract}
  \presenter{} is a \LaTeX{} presentation framework adapted to regular \LaTeX{}
  syntaxes and designed for making minimalist presentations, written in \LaTeX3.
\end{abstract}

\tableofcontents

\vfil\vfil

{

\small

GitHub Repository: \url{https://github.com/LogCreative/presenter}

You could ask usage questions in
\href{https://github.com/LogCreative/presenter/discussions}{Discussions}, report
bugs in \href{https://github.com/LogCreative/presenter/issues}{Issues} or
contribute to this project in
\href{https://github.com/LogCreative/presenter/pulls}{Pull Requests}
(very welcome).

\vfil

Copyright \copyright{} 2023 Log Creative
(\href{mailto:logcreative@outlook.com}{\texttt{logcreative@outlook.com}})

This work may be distributed and/or modified under the
conditions of the \LaTeX{} Project Public License, either version 1.3
of this license or (at your option) any later version.
The latest version of this license is in
  \url{http://www.latex-project.org/lppl.txt}
and version 1.3 or later is part of all distributions of \LaTeX{}
version 2005/12/01 or later.

This work has the LPPL maintenance status `maintained'.
 
The Current Maintainer of this work is Log Creative.

}

\clearpage

\begin{documentation}

\section{Introduction}

\presenter{} is a document class (or a package) to create presentations from
regular \LaTeX{} documents without many modifications, \textit{i.e.} present
your \LaTeX{} document in a screen-friendly way.

Compared to other existing packages, \presenter{} won't need any additional
\env{frame} or \env{slide} environments to split different pages, but uses
the original sectioning commands as the natural way to split different points.

\presenter{} aims at providing a framework with the minimum code powered by
modern \LaTeX3 facilities. For further customization, the template developer
could make use of the interfaces based on the \pkg{xtemplate}
package \cite{xtemplate}. \presenter{} highly recommends using the latest
\TeX{} distribution for the best user experience.

\presenter{} has the following main characteristics:

\begin{itemize}
  \item \textbf{Seamless Migration.} The user should feel seamless on migrating
        from document class like \cls{article}, \cls{report} to \cls{presenter}
        without modifying many structures.
  \item \textbf{Stay On Top.} Sectioning information will stay on the top of the
        slides.
  \item \textbf{Minimal Design.} The minimal design is for catering to the
        current trend without introducing too many complexities.
\end{itemize}

\textbf{
  \presenter{} is currently in ALPHA release stage, further modifications to
  the interfaces could be made without the deprecation transition stage. See
  Changes section for details.
}

\section{Basics}

Using \cls{presenter} is very simple, just swap the line of
\verb"\documentclass{article}" into \verb"\documentclass{presenter}" and you are
almost done.

\begin{center}
\begin{minipage}[c]{0.45\textwidth}
\begin{verbatim}
  \documentclass{presenter}
  \begin{document}
  \section{Get Started}
  Hello, world!
  \end{document}
\end{verbatim}
\end{minipage}
\begin{minipage}[c]{0.5\textwidth}
  \previewtest{basics}
\end{minipage}
\end{center}

You can change the page style to \texttt{empty} after the \tn{clearpage} or the
sectioning commands to remove the headline and footline temporarily.
For example, you could create a thank you page in this way
(use \verb"\vspace*{5ex}" to make a vertical space):

\begin{center}
\begin{minipage}[c]{0.45\textwidth}
\begin{verbatim}
  \documentclass{presenter}
  \begin{document}
  % previous contents ...
  \clearpage
  \thispagestyle{empty}
  \vspace*{5ex}
  \begin{center}
    \Huge Thank you!
  \end{center}
  \end{document}
\end{verbatim}
\end{minipage}
\begin{minipage}[c]{0.5\textwidth}
  \previewtest{thankyou}
\end{minipage}
\end{center}

\begin{texnote}
  Currently, there are three page styles supported by \presenter{}:
  \texttt{empty} (no headline or footline, empty page for highlighting contents),
  \texttt{plain} (which is used in the part page currently, only footline)
  and \texttt{presenterpage} (a replacement of \texttt{headings}, both headline
  and footline are displayed, but it is not encouraged to explicitly set this).

  Even though the headline or the footline is removed in some page styles, the
  geometry of the page is not modified, in the other word, the position and the
  height of the body is not modified. The example could be rewritten by two
  \verb"\vspace*{\fill}" around the ``thank you'' text, but it will not lead to
  a not precise vertical centering result (a bit lower) since there is still a
  \verb"headheight" even if it is in the \verb"empty" page style. But if the
  page has a header, then it is suitable to write \verb"\vspace*{\fill}" to make
  a vertically centered text in the page body.
\end{texnote}

\section{Options}

\begin{function}[added=2023-02-04]{article, report}
  However, if you want to use \tn{chapter} level, pass the \verb"report"
  option to the document class \cls{presenter} like
  \verb"\documentclass[report]{presenter}". The default is \verb"article"
  based.

  \begin{texnote}
    It sounds not okay to use \cls{book} class as the base document class, since
    they are typically pretty long. But if you want to use it as the base class,
    or any other document class, use the package \pkg{presenter} instead.

\begin{center}
\begin{minipage}[c]{0.45\textwidth}
\begin{verbatim}
  \documentclass{book}
  \usepackage{presenter}
  \begin{document}
  \chapter{Beginning of Everything}
  This is the beginning of everything.
  \end{document}
\end{verbatim}
\end{minipage}
\begin{minipage}[c]{0.5\textwidth}
  \previewtest{package}
\end{minipage}
\end{center}
  \end{texnote}
\end{function}

\section{Templates}

\subsection{Decomposition}

\presenter{} page can be divided into two major layers: \texttt{background}
and \texttt{foreground}. There are several components inside each layer to
composite the page, as is shown in Figure~\ref{fig:tpldecomp}. These layers
are rendered in this order to overlay one after another.

\begin{figure}
\ExplSyntaxOn
\draw_begin:
\hbox_set:Nn \l_tmpa_box { \fbox{\texttt{foreground/sectioning}} }
\draw_box_use:N \l_tmpa_box
\draw_transform_shift:n { ( -4.8cm , -6cm ) }
\draw_transform_xscale:n { 0.7 }
\draw_transform_yslant:n { 0.3 }
\hbox_set:Nn \l_tmpa_box { \previewtest[page=1]{decomp} }
\draw_box_use:N \l_tmpa_box
\draw_transform_xshift:n { 4cm }
\hbox_set:Nn \l_tmpa_box { \previewtest[page=2]{decomp} }
\draw_box_use:N \l_tmpa_box
\draw_transform_xshift:n { 4cm }
\hbox_set:Nn \l_tmpa_box { \previewtest[page=3]{decomp} }
\draw_box_use:N \l_tmpa_box
\draw_end:
\ExplSyntaxOff
\caption{Page Component Decomposition}
\label{fig:tpldecomp}
\end{figure}

\begin{description}
  \item[\texttt{background}] currently contains three components: 
    \texttt{canvas}, \texttt{headline}, and \texttt{footline}.
    \begin{description}
      \item[\texttt{background/canvas}] The bottom layer.
      \item[\texttt{background/headline}] The headline background.
        If there is no headline on the page, then this layer won't get rendered.
      \item[\texttt{background/footline}] The footline background.
        If there is no footline on the page, then this layer won't get rendered. 
    \end{description}
  \item[\texttt{foreground}] currently contains four components:
    \texttt{sectioning}, \texttt{headline}, \texttt{footline} and \texttt{canvas}.
    \begin{description}
      \item[\texttt{foreground/sectioning}] A special layer handling the
        sectioning logic during processing the body. The information processed
        in this layer could be used globally (both \texttt{background} and
        \texttt{foreground}).
      \item[\texttt{foreground/headline}] The headline text.
      \item[\texttt{foreground/footline}] The footline text.
      \item[\texttt{foreground/canvas}] The top layer. 
    \end{description}
\end{description}

\presenter{} provides some built-in templates for changing the visual
presentation of these components. The following command provides the
interfaces for loading or editing them. You could visit the
\href{https://github.com/LogCreative/presenter/tree/main/testfiles}
{\texttt{testfiles/}} folder for the code example of using these templates.

\subsection{Load Templates}

\begin{function}[added=2023-03-17,updated=2023-07-14]{\LoadPresenterBackground}
  \begin{syntax}
    \cs{LoadPresenterBackground} \oarg{component comma list} \marg{style}
  \end{syntax}
  Load presenter background style with style file \file{pretbg-<style>.sty}.
  The background style selection could be viewed in Section \ref{sec:pretbg}.
  You could specify the optional argument \meta{component comma list} to
  modify one or more components. \meta{component} could be one of
  \verb"canvas", \verb"headline" and \verb"footline". This command could only
  be used in the preamble.
\end{function}

\begin{function}[added=2023-03-17,updated=2023-07-14]{\LoadPresenterForeground}
  \begin{syntax}
    \cs{LoadPresenterForeground} \oarg{component comma list} \marg{style}
  \end{syntax}
  Load presenter background style with style file \file{pretfg-<style>.sty}.
  The foreground style selection could be viewed in Section \ref{sec:pretfg}.
  You could specify the optional argument \meta{component comma list} to
  modify one or more components. \meta{component} could be one of
  \verb"headline", \verb"sectioning", \verb"footline" and \verb"canvas". This
  command could only be used in the preamble.
\end{function}

\subsection{Edit Templates}

\begin{function}[added=2023-07-11]{\EditPresenterBackground}
  \begin{syntax}
    \cs{EditPresenterBackground} \marg{component} \marg{key-value list}
  \end{syntax}
  Edit the \meta{component} by the changed properties indicated by the
  \meta{key-value list} (\textit{e.g.} \verb"stroke=true,stroke-color=blue"
  for background style \texttt{block}).
  
  The style is indicated by the last \cs{LoadPresenterBackground}. Different
  styles may have different property keys, which could be referred to in
  Section \ref{sec:pretbg}. The editing is invalid after the next
  \cs{LoadPresenterBackground} since the loading will initialize the template
  with the default values of that style.
\end{function}

\begin{function}[added=2023-07-11]{\EditPresenterForeground}
  \begin{syntax}
    \cs{EditPresenterForeground} \marg{component} \marg{key-value list}
  \end{syntax}
  Edit the \meta{component} by the changed properties indicated by the
  \meta{key-value list} (\textit{e.g.}
  \verb"parent-style=\Large,child-style=\normalsize" for foreground style
  \texttt{dual}).
  
  The style is indicated by the last \cs{LoadPresenterForeground}. Different
  styles may have different property keys, which could be referred to in
  Section \ref{sec:pretfg}. The editing is invalid after the next
  \cs{LoadPresenterForeground} since the loading will initialize the template
  with the default values of that style.
\end{function}

\end{documentation}

\CheckSum{0}

\DocInput{presenter.dtx}

\DocInput{pretsec.dtx}

\DocInput{prettpl.dtx}

\DocInput{pretbg.dtx}

\DocInput{pretfg.dtx}

\begin{thebibliography}{0}

\bibitem{susan}
Susan~K. McConnell.
\newblock Designing effective scientific presentations, 2011.
\newblock \url{https://youtu.be/Hp7Id3Yb9XQ}.

\bibitem{xu}
Xunjie Xu.
\newblock Lightweight WYSIWYG editing for math/physics presentations, 2023.
\newblock \url{https://tuna.moe/event/2023/lightweight-wysiwyg-editing/}.

\bibitem{xtemplate}
The \LaTeX{} Project.
\newblock The \pkg{xtemplate} package: Prototype document functions, 2023.
\newblock \url{https://www.ctan.org/pkg/xtemplate}.
\end{thebibliography}

\PrintChanges

\PrintIndex

\end{document}
