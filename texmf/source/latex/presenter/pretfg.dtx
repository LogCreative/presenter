% \iffalse meta-comment
% ------------------------------------------------------------------------
% Copyright (C) 2023--2024 Log Creative <logcreative@outlook.com>
%
% This work may be distributed and/or modified under the
% conditions of the LaTeX Project Public License, either version 1.3
% of this license or (at your option) any later version.
% The latest version of this license is in
%   http://www.latex-project.org/lppl.txt
% and version 1.3 or later is part of all distributions of LaTeX
% version 2005/12/01 or later.
%
% This work has the LPPL maintenance status `maintained'.
% 
% The Current Maintainer of this work is Log Creative.
% ------------------------------------------------------------------------
% \fi
%
% \iffalse
%<*package>
\NeedsTeXFormat{LaTeX2e}[2020/10/01]
\ProvidesExplPackage
%<default>  {pretfg-default}
%<dual>  {pretfg-dual}
%<cascade>  {pretfg-cascade}
%<img>  {pretfg-img}
  {2024-11-12}{0.8.0}
%<default>  {default foreground style for presenter}
%<dual>  {dual foreground style for presenter}
%<cascade>  {cascade foreground style for presenter}
%<img>  {img foreground style for presenter}
%</package>
% \fi
%
% \iffalse
%<*package>
% \fi
%
% \section{Presenter Foreground Built-In Styles}
% \label{sec:pretfg}
%
% \begin{documentation}
%
% \begin{TemplateDescription}{foreground/sectioning}{default}
% \TemplateKey{none}{---}{}{}
% \TemplateSemantics Handle sectioning marks.\\
% \previewtest[page=4]{sectioning}
%    \begin{macrocode}
%<*default>
\DeclareTemplateInterface { foreground / sectioning } { default } { 2 }
  { }
%    \end{macrocode}
% \end{TemplateDescription}
%
% \begin{TemplateDescription}{foreground/headline}{default}
% \TemplateKey{style}{tokenlist}{Style of the headline text.}{\cs{Large}}
% \TemplateSemantics Default single headline style.
%    \begin{macrocode}
\DeclareTemplateInterface { foreground / headline } { default } { 0 }
  { style : tokenlist = \Large }
%    \end{macrocode}
% \end{TemplateDescription}
%
% \begin{TemplateDescription}{foreground/footline}{default}
% \TemplateKey{style}{tokenlist}{Style of the footline text}{\cs{footnotesize}}
% \TemplateSemantics Default page number footline style.
%    \begin{macrocode}
\DeclareTemplateInterface { foreground / footline } { default } { 0 }
  { style : tokenlist = \footnotesize }
%    \end{macrocode}
% \end{TemplateDescription}
%
% \begin{TemplateDescription}{foreground/canvas}{default}
% \TemplateKey{none}{---}{}{}
% \TemplateSemantics Default foreground canvas.
%    \begin{macrocode}
\DeclareTemplateInterface { foreground / canvas } { default } { 0 } { }
%</default>
%    \end{macrocode}
% \end{TemplateDescription}
%
% \begin{TemplateDescription}{foreground/sectioning}{dual}
% \TemplateKey{none}{---}{}{}
% \TemplateSemantics Handle sectioning mark logic for displaying two levels of
% titles.\\
% \previewtest[page=4]{dual}
%    \begin{macrocode}
%<*dual>
\DeclareTemplateInterface { foreground / sectioning } { dual } { 2 }
  { }
%    \end{macrocode}
% \end{TemplateDescription}
%
% \begin{TemplateDescription}{foreground/headline}{dual}
% \TemplateKey{parent-style}{tokenlist}{Style of parent level text}{\cs{normalsize}}
% \TemplateKey{child-style}{tokenlist}{Style of main text}{\cs{Large}}
% \TemplateSemantics Display the parent level (except the empty nodes) and
% the current level titles.
%    \begin{macrocode}
\DeclareTemplateInterface { foreground / headline } { dual } { 0 }
  { 
    parent-style : tokenlist = \normalsize ,
    child-style  : tokenlist = \Large
  }
%    \end{macrocode}
% \end{TemplateDescription}
%
% \begin{TemplateDescription}{foreground/footline}{dual}
% \TemplateKey{style}{tokenlist}{Style of the footline text}{\cs{footnotesize}}
% \TemplateSemantics title is displayed in the bottom left and the page number
% is displayed in the bottom right.
%    \begin{macrocode}
\DeclareTemplateInterface { foreground / footline } { dual } { 0 }
  { style : tokenlist = \footnotesize }
%</dual>
%    \end{macrocode}
% \end{TemplateDescription}
%
% \begin{TemplateDescription}{foreground/sectioning}{cascade}
% \TemplateKey{none}{---}{}{}
% \TemplateSemantics Handle sectioning mark logic for displaying the parent
% multi-level titles and the child title.\\
% \previewtest[page=5]{cascade}
%    \begin{macrocode}
%<*cascade>
\DeclareTemplateInterface { foreground / sectioning } { cascade } { 2 }
  { }
%    \end{macrocode}
% \end{TemplateDescription}
%
% \begin{TemplateDescription}{foreground/headline}{cascade}
% \TemplateKey{parent-style}{tokenlist}{Style of parent level text}{\cs{normalsize}}
% \TemplateKey{child-style}{tokenlist}{Style of main text}{\cs{Large}}
% \TemplateKey{parent-separator}{tokenlist}{Connection tokens between different
% different levels of titles in the parent title}{\verb"\quad{}$\triangleright$\quad{}"}
% \TemplateSemantics Display all the parent level titles separated by the
% parent-separator and the current level title.
%    \begin{macrocode}
\DeclareTemplateInterface { foreground / headline } { cascade } { 0 }
  { 
    parent-style     : tokenlist = \normalsize ,
    child-style      : tokenlist = \Large ,
    parent-separator : tokenlist = { \quad{}$\triangleright$\quad{} } ,
  }
%    \end{macrocode}
% \end{TemplateDescription}
%
% \begin{TemplateDescription}{foreground/footline}{cascade}
% \TemplateKey{style}{tokenlist}{Style of the footline text}{\cs{footnotesize}}
% \TemplateSemantics author is displayed in the bottom left, title is displayed
% in the middle of bottom, and the page number is displayed in the bottom right.
%    \begin{macrocode}
\DeclareTemplateInterface { foreground / footline } { cascade } { 0 }
  { style : tokenlist = \footnotesize }
%</cascade>
%    \end{macrocode}
% \end{TemplateDescription}
%
% \begin{TemplateDescription}{foreground/canvas}{img}
% \TemplateKey{img}{tokenlist}{image insertion control sequences}{}
% \TemplateKey{pos}{choice}{one of center position \texttt{center}, or corner
%   position \texttt{top-left}, \texttt{top-right}, \texttt{bottom-left},
%   \texttt{bottom-right}}{center}
% \TemplateKey{margin-x}{length}{image margin in horizontal direction}{0cm}
% \TemplateKey{margin-y}{length}{image margin in vertical direction}{0cm}
% \TemplateSemantics Alias to template \texttt{background/canvas/img} in
% Section \ref{tpl:pretbg-img} but in the \texttt{foreground/canvas} layer.
% Use this template to make sure the \meta{img} will not be covered by the
% text in the body or other templates.\\
% \previewtest[page=3]{img}
%    \begin{macrocode}
%<*img>
\DeclareTemplateInterface { foreground / canvas } { img } { 0 }
  {
    img      : tokenlist = ,
    pos      : tokenlist = center ,
    margin-x : length    = 0cm ,
    margin-y : length    = 0cm
  }
%</img>
%    \end{macrocode}
% \end{TemplateDescription}
%
% \end{documentation}
%
% \begin{implementation}
%
% \subsection{Implementation Code}
%
% Most of the foreground styles require the sectioning information provided by
% package \pkg{pretsec}.
%    \begin{macrocode}
%<*!img>
\RequirePackage { pretsec }
%</!img>
%    \end{macrocode}
%
% \begin{texnote}
%
% The main package has already loaded \pkg{pretsec}. The reason to write the
% requirement here again is to meet the standing-free property of the package.
% And \tn{RequirePackage} designed by \LaTeX{} is smart enough to redefine those
% loaded packages to \tn{endinput}, where no overhead is introduced. 
%
% For the implementation of \texttt{foreground/sectioning}, use variable of \LaTeX3
% directly instead of the traditional mark, since the traditional mark only gets
% updated when the page is created. Sometimes a higher level of sectioning gets
% inserted and no intermediate information until the next sectioning level, and
% the information could be swallowed since it only compares the information from
% the previous page state, which is not beneficial for the new \presenter{}
% mechanism running.
%
% \end{texnote}
%
% \begin{macro}[int]{foreground/headline/default,foreground/sectioning/default,
%   foreground/footline/default,foreground/canvas/default}
% \textbf{\pkg{default} foreground style.}
%    \begin{macrocode}
%<*default>
%<@@=pretfg_default>
\DeclareTemplateCode { foreground / sectioning } { default } { 2 }
  { }
  {
    \pretsec_register_name_title:nn { #1 } { #2 }
  }
\tl_new:N \l_@@_headline_title_tl
\tl_new:N \l_@@_headline_style_tl
\tl_new:N \l_@@_cur_title_tl
\DeclareTemplateCode { foreground / headline } { default } { 0 }
  { style = \l_@@_headline_style_tl }
  {
    \AssignTemplateKeys
    \pretsec_get_cur_title:N \l_@@_cur_title_tl
    \group_begin:
      \l_@@_headline_style_tl \l_@@_cur_title_tl
    \group_end:
    \hfil
  }
\tl_new:N \l_@@_footline_style_tl
\DeclareTemplateCode { foreground / footline } { default } { 0 }
  { style = \l_@@_footline_style_tl }
  {
    \AssignTemplateKeys
    \group_begin:
      \l_@@_footline_style_tl
      \hfil
      \thepage
    \group_end:
  }
\DeclareTemplateCode { foreground / canvas } { default } { 0 } { } { }
%</default>
%    \end{macrocode}
% \end{macro}
%
% \begin{macro}[int]{foreground/headline/dual,foreground/sectioning/dual,
%   foreground/footline/dual}
% \textbf{\pkg{dual} foreground style.}
% It seems to be in horizontal mode in the header, use a \cs{vbox} to create a
% vertical mode environment in order to adjust the vertical position of the
% titles.
%
% If the default \texttt{headsep} is too small for \pkg{dual} style headline
% (overfull \cs{vbox} and weird presentation may occur), try to use
% \cs{geometry} to set both \texttt{headsep} and \texttt{top} to a bigger value
% to get a better visual presentation. Currently, this style won't set
% \cs{geometry} additionally for the decoupling reason.
%    \begin{macrocode}
%<*dual>
%<@@=pretfg_dual>
\DeclareTemplateCode { foreground / sectioning } { dual } { 2 }
  { }
  {
    \pretsec_register_name_title:nn { #1 } { #2 }
    \pretsec_push_stack_name_title:nn { #1 } { #2 }
  }
\tl_new:N \l_@@_headline_parent_style_tl
\tl_new:N \l_@@_headline_child_style_tl
\seq_new:N \l_@@_parent_seq
\tl_new:N \l_@@_parent_tl
\tl_new:N \l_@@_child_tl
\DeclareTemplateCode { foreground / headline } { dual } { 0 }
  {
    parent-style = \l_@@_headline_parent_style_tl ,
    child-style  = \l_@@_headline_child_style_tl
  }
  {
    \AssignTemplateKeys
    \group_begin:
    \pretsec_get_stack_parent_child_title:NN \l_@@_parent_seq
      \l_@@_child_tl
    \seq_if_empty:NTF \l_@@_parent_seq
      {
        \l_@@_headline_child_style_tl \l_@@_child_tl
      }
      {
        \seq_get:NN \l_@@_parent_seq \l_@@_parent_tl
        \vbox:n
          {
            \l_@@_headline_parent_style_tl \l_@@_parent_tl
            \skip_vertical:N \c_zero_dim
            \l_@@_headline_child_style_tl \l_@@_child_tl
          }
      }
    \group_end:
    \hfil
  }
\tl_new:N \l_@@_footline_style_tl
\DeclareTemplateCode { foreground / footline } { dual } { 0 }
  { style = \l_@@_footline_style_tl }
  {
    \AssignTemplateKeys
    \group_begin:
      \l_@@_footline_style_tl
      \@title
      \hfil
      \thepage
    \group_end:
  }
%</dual>
%    \end{macrocode}
% \end{macro}
%
% \begin{macro}[int]{foreground/headline/cascade,foreground/sectioning/cascade,
%   foreground/footline/cascade}
% \textbf{\pkg{cascade} foreground style.}
% Almost the same as \pkg{dual} style. But all the parent nodes are displayed.
% Since the parent sequence returned from \cs{pretsec_get_stack_parent_child_title:NN}
% is in the reversed order, when displaying it, reverese the sequence first.
%
%    \begin{macrocode}
%<*cascade>
%<@@=pretfg_cascade>
\DeclareTemplateCode { foreground / sectioning } { cascade } { 2 }
  { }
  {
    \pretsec_register_name_title:nn { #1 } { #2 }
    \pretsec_push_stack_name_title:nn { #1 } { #2 }
  }
\tl_new:N \l_@@_headline_parent_style_tl
\tl_new:N \l_@@_headline_child_style_tl
\tl_new:N \l_@@_headline_parent_separator_tl
\seq_new:N \l_@@_parent_seq
\tl_new:N \l_@@_child_tl
\DeclareTemplateCode { foreground / headline } { cascade } { 0 }
  {
    parent-style     = \l_@@_headline_parent_style_tl ,
    child-style      = \l_@@_headline_child_style_tl ,
    parent-separator = \l_@@_headline_parent_separator_tl
  }
  {
    \AssignTemplateKeys
    \group_begin:
    \pretsec_get_stack_parent_child_title:NN \l_@@_parent_seq
      \l_@@_child_tl
    \seq_if_empty:NTF \l_@@_parent_seq
      {
        \l_@@_headline_child_style_tl \l_@@_child_tl
      }
      {
        \seq_reverse:N \l_@@_parent_seq
        \vbox:n
          {
            \l_@@_headline_parent_style_tl
            \seq_use:Nn \l_@@_parent_seq
              { \l_@@_headline_parent_separator_tl }
            \skip_vertical:N \c_zero_dim
            \l_@@_headline_child_style_tl \l_@@_child_tl
          }
      }
    \group_end:
    \hfil
  }
\tl_new:N \l_@@_footline_style_tl
\DeclareTemplateCode { foreground / footline } { cascade } { 0 }
  { style = \l_@@_footline_style_tl }
  {
    \AssignTemplateKeys
    \group_begin:
      \l_@@_footline_style_tl
      \@author
      \hfil
      \@title
      \hfil
      \thepage
    \group_end:
  }
%</cascade>
%    \end{macrocode}
% \end{macro}
%
% \begin{macro}[int]{foreground/canvas/img}
% \textbf{\pkg{img} foreground style.}
% This style is an alias to \texttt{background/canvas/img}.
% \begin{texnote}
% The use of \cs{UseTemplate} here does not under \cs{exp_args} since the
% special parsing of \verb"choice" type key could not take that.
% \end{texnote}
%    \begin{macrocode}
%<*img>
%<@@=pretfg_img>
\RequirePackage { pretbg-img }
\tl_new:N \l_@@_canvas_img_tl
\tl_new:N \l_@@_canvas_pos_tl
\dim_new:N \l_@@_canvas_margin_x_dim
\dim_new:N \l_@@_canvas_margin_y_dim
\DeclareTemplateCode { foreground / canvas } { img } { 0 }
  {
    img      = \l_@@_canvas_img_tl ,
    pos      = \l_@@_canvas_pos_tl ,
    margin-x = \l_@@_canvas_margin_x_dim ,
    margin-y = \l_@@_canvas_margin_y_dim
  }
  {
    \AssignTemplateKeys
    \UseTemplate { background / canvas } { img }
      {
        img      = \l_@@_canvas_img_tl ,
        pos      = \l_@@_canvas_pos_tl ,
        margin-x = \l_@@_canvas_margin_x_dim ,
        margin-y = \l_@@_canvas_margin_y_dim
      }
  }
%</img>
%    \end{macrocode}
% \end{macro}
%
% \end{implementation}
%
% \iffalse
%</package>
% \fi
%
\endinput
