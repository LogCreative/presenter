% \iffalse meta-comment
% ------------------------------------------------------------------------
% Copyright (C) 2023--2024 Log Creative <logcreative@outlook.com>
%
% This work may be distributed and/or modified under the
% conditions of the LaTeX Project Public License, either version 1.3
% of this license or (at your option) any later version.
% The latest version of this license is in
%   http://www.latex-project.org/lppl.txt
% and version 1.3 or later is part of all distributions of LaTeX
% version 2005/12/01 or later.
%
% This work has the LPPL maintenance status `maintained'.
% 
% The Current Maintainer of this work is Log Creative.
% ------------------------------------------------------------------------
% \fi
%
% \begin{implementation}
%
% \iffalse
%<*package>
\NeedsTeXFormat{LaTeX2e}[2020/10/01]
\ProvidesExplPackage{prettpl}{2024-01-17}{0.7.2}
  {template management for presenter}
%</package>
% \fi
%
% \iffalse
%<*package>
% \fi
%
% \section{Presenter Template}
% \label{sec:prettpl}
%
% \pkg{prettpl} package contains code for template management.
%
%    \begin{macrocode}
%<@@=prettpl>
%    \end{macrocode}
%
% \subsection{Template Management Helper}
%
% Since templates has already been integrated into the \LaTeXe{} kernel since
% 2024-06-01, \pkg{xtemplate} package is no longer need. And new interface
% \cs{NewTemplateType} replaces the old interface \cs{NewTemplateType}.
%
% \begin{macro}[TF]{\prettpl_if_template_exist:nn}
% If the \meta{template type} in style \meta{template name} exists for the
% current loaded code.
% \begin{arguments}
%   \item Template type
%   \item Template name
% \end{arguments}
%    \begin{macrocode}
\prg_new_conditional:Npnn \prettpl_if_template_exist:nn #1#2 { T, F, TF }
{
  \cs_if_exist:cTF { \c__template_code_root_tl #1 / #2 }
    { \prg_return_true: }
    { \prg_return_false: }
}
%    \end{macrocode}
% \begin{texnote}
% Package \pkg{lttemplate} does not provide a user interface like
% \cs{IfInstanceExistTF} to detect whether the template exists. And the internal
% interface \cs{__template_execute_if_code_exist:nnT} will raise the error if
% the template code does not exist (can be seen as only \texttt{T} conditional
% exists). This macro is referred to the code of the internal interface but
% won't raise error if it does not exist.
%
% \pkg{xtemplate} old collection model is suitable for the use of \presenter{}, 
% which may be the reason why the user interface for detecting template does
% not exist. But the collection model has been removed from \pkg{lttemplate}
% and \presenter{} has to use a cutomized interface for the use of the package
% \pkg{lttemplate} at present.
% \end{texnote}
% \end{macro}
%
% \begin{macro}{\prettpl_load_template:nn}
% Load the \meta{template type} in style \meta{template name}. Mainly redeclare
% the instance \texttt{base} to the new template.
% If the template code does not exists, then an info message will be recorded
% in the log file.
% \begin{arguments}
%   \item Template type
%   \item Template name
% \end{arguments}
%    \begin{macrocode}
\msg_new:nnn { presenter } { no-template }
  { Object~ type~ `#1'~ in~ style~ `#2'~ does~ not~ exist. }
\cs_new:Npn \prettpl_load_template:nn #1#2
  {
    \prettpl_if_template_exist:nnTF { #1 } { #2 }
      { \DeclareInstance { #1 } { base } { #2 } { } }
      { \msg_info:nnnn { presenter } { no-template } { #1 } { #2 } }
  }
%    \end{macrocode}
% \end{macro}
%
% \subsection{Object Type Declaration}
%
% Declare template types.
% \begin{macro}{background/canvas}
% The background bottom layer.
%    \begin{macrocode}
\NewTemplateType { background / canvas } { 0 }
%    \end{macrocode}
% \end{macro}
% \begin{macro}{background/headline}
% The background rendering of the headline, which is hooked to \tn{@oddhead}.
%    \begin{macrocode}
\NewTemplateType { background / headline } { 0 }
%    \end{macrocode}
% \end{macro}
% \begin{macro}{background/footline}
% The background rendering of the footline, which is hooked to \tn{@oddfoot}.
%    \begin{macrocode}
\NewTemplateType { background / footline } { 0 }
%    \end{macrocode}
% \end{macro}
% \begin{macro}{foreground/sectioning}
% Handle the sectioning logic. Use in pair with \texttt{foreground/headline}
% (or \texttt{foreground/footline} if you want to show the sectioning
% information in the footline).
% The call of \cs{pretsec_register_name_title:nn} is required for every
% implementation code of this template type.
% \begin{arguments}
%   \item the name of sectioning.
%   \item the content of the sectioning title.
% \end{arguments}
%    \begin{macrocode}
\NewTemplateType { foreground / sectioning } { 2 }
%    \end{macrocode}
% \end{macro}
% \begin{macro}{foreground/headline}
% The foreground display method of the headline.
%    \begin{macrocode}
\NewTemplateType { foreground / headline } { 0 }
%    \end{macrocode}
% \end{macro}
% \begin{macro}{foreground/footline}
% The foreground display of the footline.
%    \begin{macrocode}
\NewTemplateType { foreground / footline } { 0 }
%    \end{macrocode}
% \end{macro}
% \begin{macro}{foreground/canvas}
% The foreground top layer.
%    \begin{macrocode}
\NewTemplateType { foreground / canvas } { 0 }
%    \end{macrocode}
% \end{macro}
%
% \subsection{User Interface}
%
% \begin{macro}{\LoadPresenterBackground}
% Load the style file \file{pretbg-#2.sty} and set the component template based
% on the first argument (could be a comma list).
% \begin{arguments}
%   \item component comma list.
%         The default value will set up all components of this style.
%   \item background style name.
% \end{arguments}
%    \begin{macrocode}
\NewDocumentCommand { \LoadPresenterBackground }
  { O{canvas, headline, footline} m }
  {
    \RequirePackage { pretbg-#2 }
    \clist_map_inline:nn { #1 } {
      \prettpl_load_template:nn { background / ##1 } { #2 }
    }
  }
%    \end{macrocode}
% \end{macro}
%
% \begin{macro}{\LoadPresenterForeground}
% Load the style file \file{pretfg-#2.sty} and set the component template based
% on the first argument (could be a comma list).
% \begin{arguments}
%   \item component comma list.
%         The default value will set up all components of this style.
%   \item foreground style name.
% \end{arguments}
%    \begin{macrocode}
\NewDocumentCommand { \LoadPresenterForeground } 
  { O{headline, sectioning, footline, canvas} m }
  {
    \RequirePackage { pretfg-#2 }
    \clist_map_inline:nn { #1 } {
      \prettpl_load_template:nn { foreground / ##1 } { #2 }
    }
  }
%    \end{macrocode}
% \end{macro}
%
% \begin{macro}{\EditPresenterBackground}
% Edit the background component \meta{component name} on instance \texttt{base}
% by the property list \meta{key-value}.
% \begin{arguments}
%   \item component name
%   \item key-value property list
% \end{arguments}
%    \begin{macrocode}
\NewDocumentCommand { \EditPresenterBackground } { m m } {
  \EditInstance { background / #1 } { base } { #2 }
}
%    \end{macrocode}
% \end{macro}
%
% \begin{macro}{\EditPresenterForeground}
% Edit the foreground component \meta{component name} on instance \texttt{base}
% by the property list \meta{key-value}.
% \begin{arguments}
%   \item component name
%   \item key-value property list
% \end{arguments}
%    \begin{macrocode}
\NewDocumentCommand { \EditPresenterForeground } { m m } {
  \EditInstance { foreground / #1 } { base } { #2 }
}
%    \end{macrocode}
% \end{macro}
%
% \iffalse
%</package>
% \fi
%
% \end{implementation}
\endinput
